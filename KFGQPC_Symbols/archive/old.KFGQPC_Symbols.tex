% !TeX program = xelatex
% http://fonts.qurancomplex.gov.sa/?page_id=486
\documentclass{scrartcl}
% ===================
%     PACKAGES
% ===================
%\usepackage{pgffor}% For loops
\usepackage{longtable}
\usepackage{libertine}
\usepackage{polyglossia}
\usepackage{multicol}
\usepackage{import}
\usepackage{xltxtra}
\usepackage{url}
\usepackage[a4paper]{geometry}
\usepackage{fontspec}
\usepackage[%
  pdfauthor={Muḥammad Khālid Ḥussain},
  pdftitle={KFGQPC Arabic Symbols 01 Glyph Table},
  pdfsubject={Font Reference},
  pdfkeywords={font, KFGQPC, symbols, arabic},
  pdfstartview=FitH,
  pdfdisplaydoctitle=true,
  colorlinks=true,
  urlcolor=blue
]{hyperref}

% No indentation
\setlength{\parindent}{0in}

% ===================
%     FONTS
% ===================
\setmonofont{Inconsolata}
\newfontfamily\QPCSymbols[
  Scale=2.2,
]{KFGQPC Arabic Symbols 01}

% Polyglossia settings
\setmainlanguage{english}
\setotherlanguage[calendar=hijri, numerals=mashriq]{arabic}
\newfontfamily\arabicfont[
  Script=Arabic,
  Numbers=Proportional,
  Scale=1.4%,
]{Scheherazade}

% ===================
%     METADATA
% ===================
\author{Muḥammad Khālid Ḥussain}
\title{KFGQPC Arabic Symbols 01 Glyph Table}
\date{\Hijritoday[1]}

\begin{document}
\maketitle

\section{Introduction}

This project aims to make the glyphs from the \verb$KFGQPC Arabic Symbols 01$ font more acessible to users in XeLaTeX, while at the same time providing a higher quality reference for those using Microsoft Word.\\

As of writing, the font can be downloaded from\\
\url{http://fonts.qurancomplex.gov.sa/download/}\\

The source files for this document can be found at\\
\url{https://github.com/khalid-hussain/Latex-Utilities/tree/master/KFGQPC_Symbols}

\section{Using with XeLaTeX{}}

To use this font with XeLaTeX, define the font first. An example of such a definition is as follows.

  \begin{verbatim}
  \newfontfamily\QPCSymbols[
    Scale=2.2,
  ]{KFGQPC Arabic Symbols 01}
  \end{verbatim}

Then simply call \verb$\XeTeXglyph <number>$ to call the appropriate glyph into your document. If you wish to use the glyps in between other text, it is recommended to create a macro. An example of a macro which builds upon the last example is as follows.

  \begin{verbatim}
  \newcommand{\<your_command_name>}[1]{{\QPCSymbols{\XeTeXglyph <number>}}}
  \end{verbatim}

Do not forget to scale the glyph according to main text font, otherwise, you will have slight gaps between sentences to make up for the difference in heights of the glyph and the text.

\section{Using with Microsoft Word}

To use the glyphs in Microsoft Word, type the corresponding key and change its font to the 
`KFGQPC Arabic Symbols 01' font.\\

Credits\\
\url{ http://fonts.qurancomplex.gov.sa/?page_id=486}

\section{Glyph Table}

\begin{longtable}{c|c|c|c|c}
No & Symbol  & Arabic Text & Key & XeLaTeX\\
\hline
1 & {\QPCSymbols\XeTeXglyph 2}  & \textarabic{بسم الله الرحمن الرحيم} & \texttt{!} & \verb$\XeTeXglyph 2$  \\
\hline
2 & {\QPCSymbols\XeTeXglyph 3}  & \textarabic{بسم الله الرحمن الرحيم} & \texttt{"} & \verb$\XeTeXglyph 3$  \\
\hline
3 & {\QPCSymbols\XeTeXglyph 4}  & \textarabic{بسم الله الرحمن الرحيم} & \texttt{\#} & \verb$\XeTeXglyph 4$  \\
\hline
4 & {\QPCSymbols\XeTeXglyph 5}  & \textarabic{أسماء الله الحسنى} & \texttt{\$} & \verb$\XeTeXglyph 5$  \\
\hline
5 & {\QPCSymbols\XeTeXglyph 6}  & \textarabic{الله} & \texttt{\%} & \verb$\XeTeXglyph 6$  \\
\hline
6 & {\QPCSymbols\XeTeXglyph 7}  & \textarabic{محمد} & \texttt{\&} & \verb$\XeTeXglyph 7$  \\
\hline
7 & {\QPCSymbols\XeTeXglyph 8}  & \textarabic{قرآن كريم} & \texttt{'} & \verb$\XeTeXglyph 8$  \\
\hline
8 & {\QPCSymbols\XeTeXglyph 9}  & \textarabic{صدق الله العظيم} & \texttt{(} & \verb$\XeTeXglyph 9$  \\
\hline
9 & {\QPCSymbols\XeTeXglyph 10}  & \textarabic{كل عام وأنتم بخير} & \texttt{)} & \verb$\XeTeXglyph 10$  \\
\hline
10 & {\QPCSymbols\XeTeXglyph 11}  & \textarabic{السبت} & \texttt{*} & \verb$\XeTeXglyph 11$  \\
\hline
11 & {\QPCSymbols\XeTeXglyph 12}  & \textarabic{الأحد} & \texttt{+} & \verb$\XeTeXglyph 12$  \\
\hline
12 & {\QPCSymbols\XeTeXglyph 13}  & \textarabic{الاثنين} & \texttt{,} & \verb$\XeTeXglyph 13$  \\
\hline
13 & {\QPCSymbols\XeTeXglyph 14}  & \textarabic{الثلاثاء} & \texttt{-} & \verb$\XeTeXglyph 14$  \\
\hline
14 & {\QPCSymbols\XeTeXglyph 15}  & \textarabic{الأربعاء} & \texttt{.} & \verb$\XeTeXglyph 15$  \\
\hline
15 & {\QPCSymbols\XeTeXglyph 16}  & \textarabic{الخميس} & \texttt{/} & \verb$\XeTeXglyph 16$  \\
\hline
16 & {\QPCSymbols\XeTeXglyph 17}  & \textarabic{الجمعة} & \texttt{0} & \verb$\XeTeXglyph 17$  \\
\hline
17 & {\QPCSymbols\XeTeXglyph 18}  & \textarabic{محرم} & \texttt{1} & \verb$\XeTeXglyph 18$  \\
\hline
18 & {\QPCSymbols\XeTeXglyph 19}  & \textarabic{صفر} & \texttt{2} & \verb$\XeTeXglyph 19$  \\
\hline
19 & {\QPCSymbols\XeTeXglyph 20}  & \textarabic{ربيع الأول} & \texttt{3} & \verb$\XeTeXglyph 20$  \\
\hline
20 & {\QPCSymbols\XeTeXglyph 21}  & \textarabic{ربيع الثاني} & \texttt{4} & \verb$\XeTeXglyph 21$  \\
\hline
21 & {\QPCSymbols\XeTeXglyph 22}  & \textarabic{جمادى الأولى} & \texttt{5} & \verb$\XeTeXglyph 22$  \\
\hline
22 & {\QPCSymbols\XeTeXglyph 23}  & \textarabic{جمادى الآخرة} & \texttt{6} & \verb$\XeTeXglyph 23$  \\
\hline
23 & {\QPCSymbols\XeTeXglyph 24}  & \textarabic{رجب} & \texttt{7} & \verb$\XeTeXglyph 24$  \\
\hline
24 & {\QPCSymbols\XeTeXglyph 25}  & \textarabic{شعبان} & \texttt{8} & \verb$\XeTeXglyph 25$  \\
\hline
25 & {\QPCSymbols\XeTeXglyph 26}  & \textarabic{رمضان} & \texttt{9} & \verb$\XeTeXglyph 26$  \\
\hline
\end{longtable}

\begin{tabular}{c|c|c|c|c}
No & Symbol  & Arabic Text & Key & XeLaTeX\\
\hline
26 & {\QPCSymbols\XeTeXglyph 27}  & \textarabic{شوال} & \texttt{:} & \verb$\XeTeXglyph 27$  \\
\hline
27 & {\QPCSymbols\XeTeXglyph 28}  & \textarabic{ذو القعدة} & \texttt{;} & \verb$\XeTeXglyph 28$  \\
\hline
28 & {\QPCSymbols\XeTeXglyph 29}  & \textarabic{ذو الحجة} & \texttt{<} & \verb$\XeTeXglyph 29$  \\
\hline
29 & {\QPCSymbols\XeTeXglyph 30}  & \textarabic{عيد مبارك} & \texttt{=} & \verb$\XeTeXglyph 30$  \\
\hline
30 & {\QPCSymbols\XeTeXglyph 31}  & \textarabic{عيد سعيد} & \texttt{>} & \verb$\XeTeXglyph 31$  \\
\hline
31 & {\QPCSymbols\XeTeXglyph 32}  & \textarabic{المحتويات} & \texttt{?} & \verb$\XeTeXglyph 32$  \\
\hline
32 & {\QPCSymbols\XeTeXglyph 33}  & \textarabic{تم بحمد الله} & \texttt{@} & \verb$\XeTeXglyph 33$  \\
\hline
33 & {\QPCSymbols\XeTeXglyph 34}  & \textarabic{الله أكبر} & \texttt{A} & \verb$\XeTeXglyph 34$  \\
\hline
34 & {\QPCSymbols\XeTeXglyph 35}  & \textarabic{جل جلاله} & \texttt{B} & \verb$\XeTeXglyph 35$  \\
\hline
35 & {\QPCSymbols\XeTeXglyph 36}  & \textarabic{جل وعلا} & \texttt{C} & \verb$\XeTeXglyph 36$  \\
\hline
36 & {\QPCSymbols\XeTeXglyph 37}  & \textarabic{عز وجل} & \texttt{D} & \verb$\XeTeXglyph 37$  \\
\hline
37 & {\QPCSymbols\XeTeXglyph 38}  & \textarabic{سبحانه وتعالى} & \texttt{E} & \verb$\XeTeXglyph 38$  \\
\hline
38 & {\QPCSymbols\XeTeXglyph 39}  & \textarabic{تبارك وتعالى} & \texttt{F} & \verb$\XeTeXglyph 39$  \\
\hline
39 & {\QPCSymbols\XeTeXglyph 40}  & \textarabic{صلى الله عليه وعلى آله وسلم} & \texttt{G} & \verb$\XeTeXglyph 40$  \\
\hline
40 & {\QPCSymbols\XeTeXglyph 41}  & \textarabic{صلى الله عليه وسلم} & \texttt{H} & \verb$\XeTeXglyph 41$  \\
\hline
41 & {\QPCSymbols\XeTeXglyph 42}  & \textarabic{رضي الله عنه} & \texttt{I} & \verb$\XeTeXglyph 42$  \\
\hline
42 & {\QPCSymbols\XeTeXglyph 43}  & \textarabic{رضي الله عنها} & \texttt{J} & \verb$\XeTeXglyph 43$  \\
\hline
43 & {\QPCSymbols\XeTeXglyph 44}  & \textarabic{رضي الله عنهن} & \texttt{K} & \verb$\XeTeXglyph 44$  \\
\hline
44 & {\QPCSymbols\XeTeXglyph 45}  & \textarabic{رضي الله عنهما} & \texttt{L} & \verb$\XeTeXglyph 45$  \\
\hline
45 & {\QPCSymbols\XeTeXglyph 46}  & \textarabic{رضي الله عنهم} & \texttt{M} & \verb$\XeTeXglyph 46$  \\
\hline
46 & {\QPCSymbols\XeTeXglyph 47}  & \textarabic{عليه وعلى آله الصلاة والسلام} & \texttt{N} & \verb$\XeTeXglyph 47$  \\
\hline
47 & {\QPCSymbols\XeTeXglyph 48}  & \textarabic{عليه الصلاة والسلام} & \texttt{O} & \verb$\XeTeXglyph 48$  \\
\hline
48 & {\QPCSymbols\XeTeXglyph 49}  & \textarabic{عليها السلام} & \texttt{P} & \verb$\XeTeXglyph 49$  \\
\hline
49 & {\QPCSymbols\XeTeXglyph 50}  & \textarabic{عليهم السلام} & \texttt{Q} & \verb$\XeTeXglyph 50$  \\
\hline
50 & {\QPCSymbols\XeTeXglyph 51}  & \textarabic{عليهما السلام} & \texttt{R} & \verb$\XeTeXglyph 51$  \\
\hline
\end{tabular}

\begin{tabular}{c|c|c|c|c}
No & Symbol  & Arabic Text & Key & XeLaTeX\\
\hline
51 & {\QPCSymbols\XeTeXglyph 52}  & \textarabic{عليه السلام} & \texttt{S} & \verb$\XeTeXglyph 52$  \\
\hline
52 & {\QPCSymbols\XeTeXglyph 53}  & \textarabic{رحمها الله} & \texttt{T} & \verb$\XeTeXglyph 53$  \\
\hline
53 & {\QPCSymbols\XeTeXglyph 54}  & \textarabic{رحمهن الله} & \texttt{U} & \verb$\XeTeXglyph 54$  \\
\hline
54 & {\QPCSymbols\XeTeXglyph 55}  & \textarabic{رحمه الله} & \texttt{V} & \verb$\XeTeXglyph 55$  \\
\hline
55 & {\QPCSymbols\XeTeXglyph 56}  & \textarabic{رحمهما الله} & \texttt{W} & \verb$\XeTeXglyph 56$  \\
\hline
56 & {\QPCSymbols\XeTeXglyph 57}  & \textarabic{رحمهم الله} & \texttt{X} & \verb$\XeTeXglyph 57$  \\
\hline
57 & {\QPCSymbols\XeTeXglyph 58}  & \textarabic{مقدمة} & \texttt{Y} & \verb$\XeTeXglyph 58$  \\
\hline
58 & {\QPCSymbols\XeTeXglyph 59}  & \textarabic{فهرس} & \texttt{Z} & \verb$\XeTeXglyph 59$  \\
\hline
59 & {\QPCSymbols\XeTeXglyph 60}  & \textarabic{الفصل} & \texttt{[} & \verb$\XeTeXglyph 60$  \\
\hline
60 & {\QPCSymbols\XeTeXglyph 61}  & \textarabic{تمهيد} & \texttt{ } & \verb$\XeTeXglyph 61$  \\
\hline
61 & {\QPCSymbols\XeTeXglyph 62}  & \textarabic{تمت} & \texttt{]} & \verb$\XeTeXglyph 62$  \\
\hline
62 & {\QPCSymbols\XeTeXglyph 63}  & \textarabic{الباب} & \texttt{\^} & \verb$\XeTeXglyph 63$  \\
\hline
63 & {\QPCSymbols\XeTeXglyph 64}  & \textarabic{الجزء} & \texttt{\_} & \verb$\XeTeXglyph 64$  \\
\hline
64 & {\QPCSymbols\XeTeXglyph 65}  & \textarabic{تهانينا} & \texttt{`} & \verb$\XeTeXglyph 65$  \\
\hline
65 & {\QPCSymbols\XeTeXglyph 66}  & \textarabic{الله أكبر} & \texttt{a} & \verb$\XeTeXglyph 66$  \\
\hline
66 & {\QPCSymbols\XeTeXglyph 67}  & \textarabic{عز وجل} & \texttt{b} & \verb$\XeTeXglyph 67$  \\
\hline
67 & {\QPCSymbols\XeTeXglyph 68}  & \textarabic{سبحانه وتعالى} & \texttt{c} & \verb$\XeTeXglyph 68$  \\
\hline
68 & {\QPCSymbols\XeTeXglyph 69}  & \textarabic{جل جلاله} & \texttt{d} & \verb$\XeTeXglyph 69$  \\
\hline
69 & {\QPCSymbols\XeTeXglyph 70}  & \textarabic{جل وعلا} & \texttt{e} & \verb$\XeTeXglyph 70$  \\
\hline
70 & {\QPCSymbols\XeTeXglyph 71}  & \textarabic{تبارك وتعالى} & \texttt{f} & \verb$\XeTeXglyph 71$  \\
\hline
71 & {\QPCSymbols\XeTeXglyph 72}  & \textarabic{صلى الله عليه وسلم} & \texttt{g} & \verb$\XeTeXglyph 72$  \\
\hline
72 & {\QPCSymbols\XeTeXglyph 73}  & \textarabic{رضي الله عنه} & \texttt{h} & \verb$\XeTeXglyph 73$  \\
\hline
73 & {\QPCSymbols\XeTeXglyph 74}  & \textarabic{رضي الله عنها} & \texttt{i} & \verb$\XeTeXglyph 74$  \\
\hline
74 & {\QPCSymbols\XeTeXglyph 75}  & \textarabic{رضي الله عنهم} & \texttt{j} & \verb$\XeTeXglyph 75$  \\
\hline
75 & {\QPCSymbols\XeTeXglyph 76}  & \textarabic{رضي الله عنهما} & \texttt{k} & \verb$\XeTeXglyph 76$  \\
\hline
\end{tabular}

\begin{tabular}{c|c|c|c|c}
No & Symbol  & Arabic Text & Key & XeLaTeX\\
\hline
76 & {\QPCSymbols\XeTeXglyph 77}  & \textarabic{رضي الله عنهن} & \texttt{l} & \verb$\XeTeXglyph 77$  \\
\hline
77 & {\QPCSymbols\XeTeXglyph 78}  & \textarabic{عليه الصلاة والسلام} & \texttt{m} & \verb$\XeTeXglyph 78$  \\
\hline
78 & {\QPCSymbols\XeTeXglyph 79}  & \textarabic{عليه السلام} & \texttt{n} & \verb$\XeTeXglyph 79$  \\
\hline
79 & {\QPCSymbols\XeTeXglyph 80}  & \textarabic{عليها السلام} & \texttt{o} & \verb$\XeTeXglyph 80$  \\
\hline
80 & {\QPCSymbols\XeTeXglyph 81}  & \textarabic{عليهم السلام} & \texttt{p} & \verb$\XeTeXglyph 81$  \\
\hline
81 & {\QPCSymbols\XeTeXglyph 82}  & \textarabic{عليهما السلام} & \texttt{q} & \verb$\XeTeXglyph 82$  \\
\hline
82 & {\QPCSymbols\XeTeXglyph 83}  & \textarabic{رحمه الله} & \texttt{r} & \verb$\XeTeXglyph 83$  \\
\hline
83 & {\QPCSymbols\XeTeXglyph 84}  & \textarabic{رحمهم الله} & \texttt{s} & \verb$\XeTeXglyph 84$  \\
\hline
84 & {\QPCSymbols\XeTeXglyph 85}  & \textarabic{رحمهما الله} & \texttt{t} & \verb$\XeTeXglyph 85$  \\
\hline
85 & {\QPCSymbols\XeTeXglyph 86}  & \textarabic{رحمها الله} & \texttt{u} & \verb$\XeTeXglyph 86$  \\
\hline
86 & {\QPCSymbols\XeTeXglyph 87}  & \textarabic{رحمهن الله} & \texttt{v} & \verb$\XeTeXglyph 87$  \\
\hline
87 & {\QPCSymbols\XeTeXglyph 88}  & \textarabic{ريال} & \texttt{w} & \verb$\XeTeXglyph 88$  \\
\hline
88 & {\QPCSymbols\XeTeXglyph 89}  & \textarabic{أوقية} & \texttt{x} & \verb$\XeTeXglyph 89$  \\
\hline
89 & {\QPCSymbols\XeTeXglyph 90}  & \textarabic{شلن} & \texttt{y} & \verb$\XeTeXglyph 90$  \\
\hline
90 & {\QPCSymbols\XeTeXglyph 91}  & \textarabic{جنيه} & \texttt{z} & \verb$\XeTeXglyph 91$  \\
\hline
91 & {\QPCSymbols\XeTeXglyph 92}  & \textarabic{درهم} & \texttt{\{} & \verb$\XeTeXglyph 92$  \\
\hline
92 & {\QPCSymbols\XeTeXglyph 93}  & \textarabic{دينار} & \texttt{|} & \verb$\XeTeXglyph 93$  \\
\hline
93 & {\QPCSymbols\XeTeXglyph 94}  & \textarabic{ليرة} & \texttt{\}} & \verb$\XeTeXglyph 94$  \\
\hline
94 & {\QPCSymbols\XeTeXglyph 95}  & \textarabic{الحزب} & \texttt{~} & \verb$\XeTeXglyph 95$  \\
\hline
95 & {\QPCSymbols\XeTeXglyph 96}  & \textarabic{الربع} & \texttt{ƒ} & \verb$\XeTeXglyph 96$  \\
\hline
96 & {\QPCSymbols\XeTeXglyph 97}  & \textarabic{الثمن} & \texttt{„} & \verb$\XeTeXglyph 97$  \\
\hline
\end{tabular}

%\begin{multicols}{3}
%  \begin{enumerate}
%    \foreach \x in {2,3,...,97}{
%      \item {\QPCSymbols\XeTeXglyph \x} \\[14pt]
%    }
%  \end{enumerate}
%\end{multicols}

\end{document}