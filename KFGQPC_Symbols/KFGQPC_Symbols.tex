% !TeX program = xelatex
% http://fonts.qurancomplex.gov.sa/?page_id=486
\documentclass[parskip=full, oneside]{article}
\setlength{\parindent}{0in}
\usepackage[a4paper]{geometry}
\usepackage{changepage}
%****************************************
%      Packages
%******************************
\usepackage{array}
\usepackage{longtable, booktabs}
\usepackage{libertine}
\usepackage{polyglossia}
\usepackage{import}
\usepackage{url}
\usepackage{fontspec}
\usepackage[%
  pdfauthor={Muḥammad Khālid Ḥussain},
  pdftitle={KFGQPC Arabic Symbols 01 Glyph Table},
  pdfsubject={Font Reference},
  pdfkeywords={font, KFGQPC, symbols, arabic},
  pdfstartview=FitH,
  pdfdisplaydoctitle=true,
  colorlinks=true,
  urlcolor=blue
]{hyperref}

%****************************************
%      Fonts
%******************************
% Polyglossia
\setmainlanguage{english}
\setotherlanguage[calendar=hijri, numerals=mashriq]{arabic}

% Monospaced font
\setmonofont{Inconsolata}

% The KFGPC Arabic Symbols font
\newfontfamily\QPCSymbols[Scale=1.8]{KFGQPC Arabic Symbols 01}
\newfontfamily\QPCSymbolsBig[Scale=3]{KFGQPC Arabic Symbols 01}

% Normal Arabic font
\newfontfamily\arabicfont[%
  Script=Arabic,%
  Numbers=Proportional,%
  Scale=1.4%
]{Scheherazade}

%****************************************
%      Metadata
%******************************
\author{Muḥammad Khālid Ḥussain}
\title{KFGQPC Arabic Symbols 01 Glyph Table}
\date{\Hijritoday[1]}

\begin{document}
\maketitle

\begin{center}
\QPCSymbolsBig{\XeTeXglyph 3}
\end{center}

\section{Introduction}

This project aims to make the glyphs from the \verb$KFGQPC Arabic Symbols 01$ 
font more acessible to users in XeLaTeX, while at the same time providing a 
higher quality reference for those using Microsoft Word.\\

As of writing, the font can be downloaded from\\
\url{http://fonts.qurancomplex.gov.sa/download/}\\

The source files for this document can be found at\\
\url{https://github.com/khalid-hussain/Latex-Utilities/tree/master/KFGQPC_Symbols}

\section{Using with XeLaTeX{}}

To use this font with XeLaTeX, define the font first. An example of such a 
definition is as follows.

  \begin{verbatim}
  \newfontfamily\QPCSymbols[
    Scale=2.2,
  ]{KFGQPC Arabic Symbols 01}
  \end{verbatim}

Then simply call \verb$\XeTeXglyph <number>$ to call the appropriate glyph into 
your document. If you wish to use the glyps in between other text, it is 
recommended to create a macro. An example of a macro which builds upon the last 
example is as follows.

  \begin{verbatim}
  \newcommand{\<your_command_name>}[1]{{\QPCSymbols{\XeTeXglyph <number>}}}
  \end{verbatim}

Do not forget to scale the glyph according to main text font, otherwise, you 
will have slight gaps between sentences to make up for the difference in 
heights of the glyph and the text.

\section{Using with Microsoft Word}

To use the glyphs in Microsoft Word, type the corresponding key and change its 
font to the `KFGQPC Arabic Symbols 01' font.\\

Credits\\
\url{http://fonts.qurancomplex.gov.sa/?page_id=486}

\section{Glyph Table}

\begin{adjustwidth}{-3cm}{-3cm}
\begin{longtable}[]{@{}ccccc@{}}
\toprule
\begin{minipage}[b]{0.04\columnwidth}\centering\strut
No\strut
\end{minipage} & \begin{minipage}[b]{0.21\columnwidth}\centering\strut
Symbol\strut
\end{minipage} & \begin{minipage}[b]{0.31\columnwidth}\centering\strut
Arabic Text\strut
\end{minipage} & \begin{minipage}[b]{0.13\columnwidth}\centering\strut
Key\strut
\end{minipage} & \begin{minipage}[b]{0.18\columnwidth}\centering\strut
\XeLaTeX\strut
\end{minipage}\tabularnewline
\midrule
\endhead
\begin{minipage}[t]{0.04\columnwidth}\centering\strut
1\strut
\end{minipage} & \begin{minipage}[t]{0.21\columnwidth}\centering\strut
\QPCSymbols{\XeTeXglyph 1}\strut
\end{minipage} & \begin{minipage}[t]{0.31\columnwidth}\centering\strut
x\strut
\end{minipage} & \begin{minipage}[t]{0.13\columnwidth}\centering\strut
\texttt{!}\strut
\end{minipage} & \begin{minipage}[t]{0.18\columnwidth}\centering\strut
\verb$\XeTeXglyph 1$\strut
\end{minipage}\tabularnewline
\begin{minipage}[t]{0.04\columnwidth}\centering\strut
2\strut
\end{minipage} & \begin{minipage}[t]{0.21\columnwidth}\centering\strut
\QPCSymbols{\XeTeXglyph 2}\strut
\end{minipage} & \begin{minipage}[t]{0.31\columnwidth}\centering\strut
\textarabic{بسم الله الرحمن الرحيم}\strut
\end{minipage} & \begin{minipage}[t]{0.13\columnwidth}\centering\strut
\texttt{"}\strut
\end{minipage} & \begin{minipage}[t]{0.18\columnwidth}\centering\strut
\verb$\XeTeXglyph 2$\strut
\end{minipage}\tabularnewline
\begin{minipage}[t]{0.04\columnwidth}\centering\strut
3\strut
\end{minipage} & \begin{minipage}[t]{0.21\columnwidth}\centering\strut
\QPCSymbols{\XeTeXglyph 3}\strut
\end{minipage} & \begin{minipage}[t]{0.31\columnwidth}\centering\strut
\textarabic{بسم الله الرحمن الرحيم}\strut
\end{minipage} & \begin{minipage}[t]{0.13\columnwidth}\centering\strut
\texttt{"}\strut
\end{minipage} & \begin{minipage}[t]{0.18\columnwidth}\centering\strut
\verb$\XeTeXglyph 3$\strut
\end{minipage}\tabularnewline
\begin{minipage}[t]{0.04\columnwidth}\centering\strut
4\strut
\end{minipage} & \begin{minipage}[t]{0.21\columnwidth}\centering\strut
\QPCSymbols{\XeTeXglyph 4}\strut
\end{minipage} & \begin{minipage}[t]{0.31\columnwidth}\centering\strut
\textarabic{بسم الله الرحمن الرحيم}\strut
\end{minipage} & \begin{minipage}[t]{0.13\columnwidth}\centering\strut
\texttt{\#}\strut
\end{minipage} & \begin{minipage}[t]{0.18\columnwidth}\centering\strut
\verb$\XeTeXglyph 4$\strut
\end{minipage}\tabularnewline
\begin{minipage}[t]{0.04\columnwidth}\centering\strut
5\strut
\end{minipage} & \begin{minipage}[t]{0.21\columnwidth}\centering\strut
\QPCSymbols{\XeTeXglyph 5}\strut
\end{minipage} & \begin{minipage}[t]{0.31\columnwidth}\centering\strut
\textarabic{أسماء الله الحسنى}\strut
\end{minipage} & \begin{minipage}[t]{0.13\columnwidth}\centering\strut
\texttt{\$}\strut
\end{minipage} & \begin{minipage}[t]{0.18\columnwidth}\centering\strut
\verb$\XeTeXglyph 5$\strut
\end{minipage}\tabularnewline
\begin{minipage}[t]{0.04\columnwidth}\centering\strut
6\strut
\end{minipage} & \begin{minipage}[t]{0.21\columnwidth}\centering\strut
\QPCSymbols{\XeTeXglyph 6}\strut
\end{minipage} & \begin{minipage}[t]{0.31\columnwidth}\centering\strut
\textarabic{الله}\strut
\end{minipage} & \begin{minipage}[t]{0.13\columnwidth}\centering\strut
\texttt{\%}\strut
\end{minipage} & \begin{minipage}[t]{0.18\columnwidth}\centering\strut
\verb$\XeTeXglyph 6$\strut
\end{minipage}\tabularnewline
\begin{minipage}[t]{0.04\columnwidth}\centering\strut
7\strut
\end{minipage} & \begin{minipage}[t]{0.21\columnwidth}\centering\strut
\QPCSymbols{\XeTeXglyph 7}\strut
\end{minipage} & \begin{minipage}[t]{0.31\columnwidth}\centering\strut
\textarabic{محمد}\strut
\end{minipage} & \begin{minipage}[t]{0.13\columnwidth}\centering\strut
\texttt{\&}\strut
\end{minipage} & \begin{minipage}[t]{0.18\columnwidth}\centering\strut
\verb$\XeTeXglyph 7$\strut
\end{minipage}\tabularnewline
\begin{minipage}[t]{0.04\columnwidth}\centering\strut
8\strut
\end{minipage} & \begin{minipage}[t]{0.21\columnwidth}\centering\strut
\QPCSymbols{\XeTeXglyph 8}\strut
\end{minipage} & \begin{minipage}[t]{0.31\columnwidth}\centering\strut
\textarabic{قرآن كريم}\strut
\end{minipage} & \begin{minipage}[t]{0.13\columnwidth}\centering\strut
\texttt{'}\strut
\end{minipage} & \begin{minipage}[t]{0.18\columnwidth}\centering\strut
\verb$\XeTeXglyph 8$\strut
\end{minipage}\tabularnewline
\begin{minipage}[t]{0.04\columnwidth}\centering\strut
9\strut
\end{minipage} & \begin{minipage}[t]{0.21\columnwidth}\centering\strut
\QPCSymbols{\XeTeXglyph 9}\strut
\end{minipage} & \begin{minipage}[t]{0.31\columnwidth}\centering\strut
\textarabic{صدق الله العظيم}\strut
\end{minipage} & \begin{minipage}[t]{0.13\columnwidth}\centering\strut
\texttt{(}\strut
\end{minipage} & \begin{minipage}[t]{0.18\columnwidth}\centering\strut
\verb$\XeTeXglyph 9$\strut
\end{minipage}\tabularnewline
\begin{minipage}[t]{0.04\columnwidth}\centering\strut
10\strut
\end{minipage} & \begin{minipage}[t]{0.21\columnwidth}\centering\strut
\QPCSymbols{\XeTeXglyph 10}\strut
\end{minipage} & \begin{minipage}[t]{0.31\columnwidth}\centering\strut
\textarabic{كل عام وأنتم بخير}\strut
\end{minipage} & \begin{minipage}[t]{0.13\columnwidth}\centering\strut
\texttt{)}\strut
\end{minipage} & \begin{minipage}[t]{0.18\columnwidth}\centering\strut
\verb$\XeTeXglyph 10$\strut
\end{minipage}\tabularnewline
\begin{minipage}[t]{0.04\columnwidth}\centering\strut
11\strut
\end{minipage} & \begin{minipage}[t]{0.21\columnwidth}\centering\strut
\QPCSymbols{\XeTeXglyph 11}\strut
\end{minipage} & \begin{minipage}[t]{0.31\columnwidth}\centering\strut
\textarabic{السبت}\strut
\end{minipage} & \begin{minipage}[t]{0.13\columnwidth}\centering\strut
\texttt{*}\strut
\end{minipage} & \begin{minipage}[t]{0.18\columnwidth}\centering\strut
\verb$\XeTeXglyph 11$\strut
\end{minipage}\tabularnewline
\begin{minipage}[t]{0.04\columnwidth}\centering\strut
12\strut
\end{minipage} & \begin{minipage}[t]{0.21\columnwidth}\centering\strut
\QPCSymbols{\XeTeXglyph 12}\strut
\end{minipage} & \begin{minipage}[t]{0.31\columnwidth}\centering\strut
\textarabic{الأحد}\strut
\end{minipage} & \begin{minipage}[t]{0.13\columnwidth}\centering\strut
\texttt{+}\strut
\end{minipage} & \begin{minipage}[t]{0.18\columnwidth}\centering\strut
\verb$\XeTeXglyph 12$\strut
\end{minipage}\tabularnewline
\begin{minipage}[t]{0.04\columnwidth}\centering\strut
13\strut
\end{minipage} & \begin{minipage}[t]{0.21\columnwidth}\centering\strut
\QPCSymbols{\XeTeXglyph 13}\strut
\end{minipage} & \begin{minipage}[t]{0.31\columnwidth}\centering\strut
\textarabic{الاثنين}\strut
\end{minipage} & \begin{minipage}[t]{0.13\columnwidth}\centering\strut
\texttt{,}\strut
\end{minipage} & \begin{minipage}[t]{0.18\columnwidth}\centering\strut
\verb$\XeTeXglyph 13$\strut
\end{minipage}\tabularnewline
\begin{minipage}[t]{0.04\columnwidth}\centering\strut
14\strut
\end{minipage} & \begin{minipage}[t]{0.21\columnwidth}\centering\strut
\QPCSymbols{\XeTeXglyph 14}\strut
\end{minipage} & \begin{minipage}[t]{0.31\columnwidth}\centering\strut
\textarabic{الثلاثاء}\strut
\end{minipage} & \begin{minipage}[t]{0.13\columnwidth}\centering\strut
\texttt{-}\strut
\end{minipage} & \begin{minipage}[t]{0.18\columnwidth}\centering\strut
\verb$\XeTeXglyph 14$\strut
\end{minipage}\tabularnewline
\begin{minipage}[t]{0.04\columnwidth}\centering\strut
15\strut
\end{minipage} & \begin{minipage}[t]{0.21\columnwidth}\centering\strut
\QPCSymbols{\XeTeXglyph 15}\strut
\end{minipage} & \begin{minipage}[t]{0.31\columnwidth}\centering\strut
\textarabic{الأربعاء}\strut
\end{minipage} & \begin{minipage}[t]{0.13\columnwidth}\centering\strut
\texttt{.}\strut
\end{minipage} & \begin{minipage}[t]{0.18\columnwidth}\centering\strut
\verb$\XeTeXglyph 15$\strut
\end{minipage}\tabularnewline
\begin{minipage}[t]{0.04\columnwidth}\centering\strut
16\strut
\end{minipage} & \begin{minipage}[t]{0.21\columnwidth}\centering\strut
\QPCSymbols{\XeTeXglyph 16}\strut
\end{minipage} & \begin{minipage}[t]{0.31\columnwidth}\centering\strut
\textarabic{الخميس}\strut
\end{minipage} & \begin{minipage}[t]{0.13\columnwidth}\centering\strut
\texttt{/}\strut
\end{minipage} & \begin{minipage}[t]{0.18\columnwidth}\centering\strut
\verb$\XeTeXglyph 16$\strut
\end{minipage}\tabularnewline
\begin{minipage}[t]{0.04\columnwidth}\centering\strut
17\strut
\end{minipage} & \begin{minipage}[t]{0.21\columnwidth}\centering\strut
\QPCSymbols{\XeTeXglyph 17}\strut
\end{minipage} & \begin{minipage}[t]{0.31\columnwidth}\centering\strut
\textarabic{الجمعة}\strut
\end{minipage} & \begin{minipage}[t]{0.13\columnwidth}\centering\strut
\texttt{0}\strut
\end{minipage} & \begin{minipage}[t]{0.18\columnwidth}\centering\strut
\verb$\XeTeXglyph 17$\strut
\end{minipage}\tabularnewline
\begin{minipage}[t]{0.04\columnwidth}\centering\strut
18\strut
\end{minipage} & \begin{minipage}[t]{0.21\columnwidth}\centering\strut
\QPCSymbols{\XeTeXglyph 18}\strut
\end{minipage} & \begin{minipage}[t]{0.31\columnwidth}\centering\strut
\textarabic{محرم}\strut
\end{minipage} & \begin{minipage}[t]{0.13\columnwidth}\centering\strut
\texttt{1}\strut
\end{minipage} & \begin{minipage}[t]{0.18\columnwidth}\centering\strut
\verb$\XeTeXglyph 18$\strut
\end{minipage}\tabularnewline
\begin{minipage}[t]{0.04\columnwidth}\centering\strut
19\strut
\end{minipage} & \begin{minipage}[t]{0.21\columnwidth}\centering\strut
\QPCSymbols{\XeTeXglyph 19}\strut
\end{minipage} & \begin{minipage}[t]{0.31\columnwidth}\centering\strut
\textarabic{صفر}\strut
\end{minipage} & \begin{minipage}[t]{0.13\columnwidth}\centering\strut
\texttt{2}\strut
\end{minipage} & \begin{minipage}[t]{0.18\columnwidth}\centering\strut
\verb$\XeTeXglyph 19$\strut
\end{minipage}\tabularnewline
\begin{minipage}[t]{0.04\columnwidth}\centering\strut
20\strut
\end{minipage} & \begin{minipage}[t]{0.21\columnwidth}\centering\strut
\QPCSymbols{\XeTeXglyph 20}\strut
\end{minipage} & \begin{minipage}[t]{0.31\columnwidth}\centering\strut
\textarabic{ربيع الأول}\strut
\end{minipage} & \begin{minipage}[t]{0.13\columnwidth}\centering\strut
\texttt{3}\strut
\end{minipage} & \begin{minipage}[t]{0.18\columnwidth}\centering\strut
\verb$\XeTeXglyph 20$\strut
\end{minipage}\tabularnewline
\begin{minipage}[t]{0.04\columnwidth}\centering\strut
21\strut
\end{minipage} & \begin{minipage}[t]{0.21\columnwidth}\centering\strut
\QPCSymbols{\XeTeXglyph 21}\strut
\end{minipage} & \begin{minipage}[t]{0.31\columnwidth}\centering\strut
\textarabic{ربيع الثاني}\strut
\end{minipage} & \begin{minipage}[t]{0.13\columnwidth}\centering\strut
\texttt{4}\strut
\end{minipage} & \begin{minipage}[t]{0.18\columnwidth}\centering\strut
\verb$\XeTeXglyph 21$\strut
\end{minipage}\tabularnewline
\begin{minipage}[t]{0.04\columnwidth}\centering\strut
22\strut
\end{minipage} & \begin{minipage}[t]{0.21\columnwidth}\centering\strut
\QPCSymbols{\XeTeXglyph 22}\strut
\end{minipage} & \begin{minipage}[t]{0.31\columnwidth}\centering\strut
\textarabic{جمادى الأولى}\strut
\end{minipage} & \begin{minipage}[t]{0.13\columnwidth}\centering\strut
\texttt{5}\strut
\end{minipage} & \begin{minipage}[t]{0.18\columnwidth}\centering\strut
\verb$\XeTeXglyph 22$\strut
\end{minipage}\tabularnewline
\begin{minipage}[t]{0.04\columnwidth}\centering\strut
23\strut
\end{minipage} & \begin{minipage}[t]{0.21\columnwidth}\centering\strut
\QPCSymbols{\XeTeXglyph 23}\strut
\end{minipage} & \begin{minipage}[t]{0.31\columnwidth}\centering\strut
\textarabic{جمادى الآخرة}\strut
\end{minipage} & \begin{minipage}[t]{0.13\columnwidth}\centering\strut
\texttt{6}\strut
\end{minipage} & \begin{minipage}[t]{0.18\columnwidth}\centering\strut
\verb$\XeTeXglyph 23$\strut
\end{minipage}\tabularnewline
\begin{minipage}[t]{0.04\columnwidth}\centering\strut
24\strut
\end{minipage} & \begin{minipage}[t]{0.21\columnwidth}\centering\strut
\QPCSymbols{\XeTeXglyph 24}\strut
\end{minipage} & \begin{minipage}[t]{0.31\columnwidth}\centering\strut
\textarabic{رجب}\strut
\end{minipage} & \begin{minipage}[t]{0.13\columnwidth}\centering\strut
\texttt{7}\strut
\end{minipage} & \begin{minipage}[t]{0.18\columnwidth}\centering\strut
\verb$\XeTeXglyph 24$\strut
\end{minipage}\tabularnewline
\begin{minipage}[t]{0.04\columnwidth}\centering\strut
25\strut
\end{minipage} & \begin{minipage}[t]{0.21\columnwidth}\centering\strut
\QPCSymbols{\XeTeXglyph 25}\strut
\end{minipage} & \begin{minipage}[t]{0.31\columnwidth}\centering\strut
\textarabic{شعبان}\strut
\end{minipage} & \begin{minipage}[t]{0.13\columnwidth}\centering\strut
\texttt{8}\strut
\end{minipage} & \begin{minipage}[t]{0.18\columnwidth}\centering\strut
\verb$\XeTeXglyph 25$\strut
\end{minipage}\tabularnewline
\begin{minipage}[t]{0.04\columnwidth}\centering\strut
26\strut
\end{minipage} & \begin{minipage}[t]{0.21\columnwidth}\centering\strut
\QPCSymbols{\XeTeXglyph 26}\strut
\end{minipage} & \begin{minipage}[t]{0.31\columnwidth}\centering\strut
\textarabic{رمضان}\strut
\end{minipage} & \begin{minipage}[t]{0.13\columnwidth}\centering\strut
\texttt{9}\strut
\end{minipage} & \begin{minipage}[t]{0.18\columnwidth}\centering\strut
\verb$\XeTeXglyph 26$\strut
\end{minipage}\tabularnewline
\begin{minipage}[t]{0.04\columnwidth}\centering\strut
27\strut
\end{minipage} & \begin{minipage}[t]{0.21\columnwidth}\centering\strut
\QPCSymbols{\XeTeXglyph 27}\strut
\end{minipage} & \begin{minipage}[t]{0.31\columnwidth}\centering\strut
\textarabic{شوال}\strut
\end{minipage} & \begin{minipage}[t]{0.13\columnwidth}\centering\strut
\texttt{:}\strut
\end{minipage} & \begin{minipage}[t]{0.18\columnwidth}\centering\strut
\verb$\XeTeXglyph 27$\strut
\end{minipage}\tabularnewline
\begin{minipage}[t]{0.04\columnwidth}\centering\strut
28\strut
\end{minipage} & \begin{minipage}[t]{0.21\columnwidth}\centering\strut
\QPCSymbols{\XeTeXglyph 28}\strut
\end{minipage} & \begin{minipage}[t]{0.31\columnwidth}\centering\strut
\textarabic{ذو القعدة}\strut
\end{minipage} & \begin{minipage}[t]{0.13\columnwidth}\centering\strut
\texttt{;}\strut
\end{minipage} & \begin{minipage}[t]{0.18\columnwidth}\centering\strut
\verb$\XeTeXglyph 28$\strut
\end{minipage}\tabularnewline
\begin{minipage}[t]{0.04\columnwidth}\centering\strut
29\strut
\end{minipage} & \begin{minipage}[t]{0.21\columnwidth}\centering\strut
\QPCSymbols{\XeTeXglyph 29}\strut
\end{minipage} & \begin{minipage}[t]{0.31\columnwidth}\centering\strut
\textarabic{ذو الحجة}\strut
\end{minipage} & \begin{minipage}[t]{0.13\columnwidth}\centering\strut
\texttt{<}\strut
\end{minipage} & \begin{minipage}[t]{0.18\columnwidth}\centering\strut
\verb$\XeTeXglyph 29$\strut
\end{minipage}\tabularnewline
\begin{minipage}[t]{0.04\columnwidth}\centering\strut
30\strut
\end{minipage} & \begin{minipage}[t]{0.21\columnwidth}\centering\strut
\QPCSymbols{\XeTeXglyph 30}\strut
\end{minipage} & \begin{minipage}[t]{0.31\columnwidth}\centering\strut
\textarabic{عيد مبارك}\strut
\end{minipage} & \begin{minipage}[t]{0.13\columnwidth}\centering\strut
\texttt{=}\strut
\end{minipage} & \begin{minipage}[t]{0.18\columnwidth}\centering\strut
\verb$\XeTeXglyph 30$\strut
\end{minipage}\tabularnewline
\begin{minipage}[t]{0.04\columnwidth}\centering\strut
31\strut
\end{minipage} & \begin{minipage}[t]{0.21\columnwidth}\centering\strut
\QPCSymbols{\XeTeXglyph 31}\strut
\end{minipage} & \begin{minipage}[t]{0.31\columnwidth}\centering\strut
\textarabic{عيد سعيد}\strut
\end{minipage} & \begin{minipage}[t]{0.13\columnwidth}\centering\strut
\texttt{>}\strut
\end{minipage} & \begin{minipage}[t]{0.18\columnwidth}\centering\strut
\verb$\XeTeXglyph 31$\strut
\end{minipage}\tabularnewline
\begin{minipage}[t]{0.04\columnwidth}\centering\strut
32\strut
\end{minipage} & \begin{minipage}[t]{0.21\columnwidth}\centering\strut
\QPCSymbols{\XeTeXglyph 32}\strut
\end{minipage} & \begin{minipage}[t]{0.31\columnwidth}\centering\strut
\textarabic{المحتويات}\strut
\end{minipage} & \begin{minipage}[t]{0.13\columnwidth}\centering\strut
\texttt{?}\strut
\end{minipage} & \begin{minipage}[t]{0.18\columnwidth}\centering\strut
\verb$\XeTeXglyph 32$\strut
\end{minipage}\tabularnewline
\begin{minipage}[t]{0.04\columnwidth}\centering\strut
33\strut
\end{minipage} & \begin{minipage}[t]{0.21\columnwidth}\centering\strut
\QPCSymbols{\XeTeXglyph 33}\strut
\end{minipage} & \begin{minipage}[t]{0.31\columnwidth}\centering\strut
\textarabic{تم بحمد الله}\strut
\end{minipage} & \begin{minipage}[t]{0.13\columnwidth}\centering\strut
\texttt{@}\strut
\end{minipage} & \begin{minipage}[t]{0.18\columnwidth}\centering\strut
\verb$\XeTeXglyph 33$\strut
\end{minipage}\tabularnewline
\begin{minipage}[t]{0.04\columnwidth}\centering\strut
34\strut
\end{minipage} & \begin{minipage}[t]{0.21\columnwidth}\centering\strut
\QPCSymbols{\XeTeXglyph 34}\strut
\end{minipage} & \begin{minipage}[t]{0.31\columnwidth}\centering\strut
\textarabic{الله أكبر}\strut
\end{minipage} & \begin{minipage}[t]{0.13\columnwidth}\centering\strut
\texttt{A}\strut
\end{minipage} & \begin{minipage}[t]{0.18\columnwidth}\centering\strut
\verb$\XeTeXglyph 34$\strut
\end{minipage}\tabularnewline
\begin{minipage}[t]{0.04\columnwidth}\centering\strut
35\strut
\end{minipage} & \begin{minipage}[t]{0.21\columnwidth}\centering\strut
\QPCSymbols{\XeTeXglyph 35}\strut
\end{minipage} & \begin{minipage}[t]{0.31\columnwidth}\centering\strut
\textarabic{جل جلاله}\strut
\end{minipage} & \begin{minipage}[t]{0.13\columnwidth}\centering\strut
\texttt{B}\strut
\end{minipage} & \begin{minipage}[t]{0.18\columnwidth}\centering\strut
\verb$\XeTeXglyph 35$\strut
\end{minipage}\tabularnewline
\begin{minipage}[t]{0.04\columnwidth}\centering\strut
36\strut
\end{minipage} & \begin{minipage}[t]{0.21\columnwidth}\centering\strut
\QPCSymbols{\XeTeXglyph 36}\strut
\end{minipage} & \begin{minipage}[t]{0.31\columnwidth}\centering\strut
\textarabic{جل وعلا}\strut
\end{minipage} & \begin{minipage}[t]{0.13\columnwidth}\centering\strut
\texttt{C}\strut
\end{minipage} & \begin{minipage}[t]{0.18\columnwidth}\centering\strut
\verb$\XeTeXglyph 36$\strut
\end{minipage}\tabularnewline
\begin{minipage}[t]{0.04\columnwidth}\centering\strut
37\strut
\end{minipage} & \begin{minipage}[t]{0.21\columnwidth}\centering\strut
\QPCSymbols{\XeTeXglyph 37}\strut
\end{minipage} & \begin{minipage}[t]{0.31\columnwidth}\centering\strut
\textarabic{عز وجل}\strut
\end{minipage} & \begin{minipage}[t]{0.13\columnwidth}\centering\strut
\texttt{D}\strut
\end{minipage} & \begin{minipage}[t]{0.18\columnwidth}\centering\strut
\verb$\XeTeXglyph 37$\strut
\end{minipage}\tabularnewline
\begin{minipage}[t]{0.04\columnwidth}\centering\strut
38\strut
\end{minipage} & \begin{minipage}[t]{0.21\columnwidth}\centering\strut
\QPCSymbols{\XeTeXglyph 38}\strut
\end{minipage} & \begin{minipage}[t]{0.31\columnwidth}\centering\strut
\textarabic{سبحانه وتعالى}\strut
\end{minipage} & \begin{minipage}[t]{0.13\columnwidth}\centering\strut
\texttt{E}\strut
\end{minipage} & \begin{minipage}[t]{0.18\columnwidth}\centering\strut
\verb$\XeTeXglyph 38$\strut
\end{minipage}\tabularnewline
\begin{minipage}[t]{0.04\columnwidth}\centering\strut
39\strut
\end{minipage} & \begin{minipage}[t]{0.21\columnwidth}\centering\strut
\QPCSymbols{\XeTeXglyph 39}\strut
\end{minipage} & \begin{minipage}[t]{0.31\columnwidth}\centering\strut
\textarabic{تبارك وتعالى}\strut
\end{minipage} & \begin{minipage}[t]{0.13\columnwidth}\centering\strut
\texttt{F}\strut
\end{minipage} & \begin{minipage}[t]{0.18\columnwidth}\centering\strut
\verb$\XeTeXglyph 39$\strut
\end{minipage}\tabularnewline
\begin{minipage}[t]{0.04\columnwidth}\centering\strut
40\strut
\end{minipage} & \begin{minipage}[t]{0.21\columnwidth}\centering\strut
\QPCSymbols{\XeTeXglyph 40}\strut
\end{minipage} & \begin{minipage}[t]{0.31\columnwidth}\centering\strut
\textarabic{صلى الله عليه وعلى آله وسلم}\strut
\end{minipage} & \begin{minipage}[t]{0.13\columnwidth}\centering\strut
\texttt{G}\strut
\end{minipage} & \begin{minipage}[t]{0.18\columnwidth}\centering\strut
\verb$\XeTeXglyph 40$\strut
\end{minipage}\tabularnewline
\begin{minipage}[t]{0.04\columnwidth}\centering\strut
41\strut
\end{minipage} & \begin{minipage}[t]{0.21\columnwidth}\centering\strut
\QPCSymbols{\XeTeXglyph 41}\strut
\end{minipage} & \begin{minipage}[t]{0.31\columnwidth}\centering\strut
\textarabic{صلى الله عليه وسلم}\strut
\end{minipage} & \begin{minipage}[t]{0.13\columnwidth}\centering\strut
\texttt{H}\strut
\end{minipage} & \begin{minipage}[t]{0.18\columnwidth}\centering\strut
\verb$\XeTeXglyph 41$\strut
\end{minipage}\tabularnewline
\begin{minipage}[t]{0.04\columnwidth}\centering\strut
42\strut
\end{minipage} & \begin{minipage}[t]{0.21\columnwidth}\centering\strut
\QPCSymbols{\XeTeXglyph 42}\strut
\end{minipage} & \begin{minipage}[t]{0.31\columnwidth}\centering\strut
\textarabic{رضي الله عنه}\strut
\end{minipage} & \begin{minipage}[t]{0.13\columnwidth}\centering\strut
\texttt{I}\strut
\end{minipage} & \begin{minipage}[t]{0.18\columnwidth}\centering\strut
\verb$\XeTeXglyph 42$\strut
\end{minipage}\tabularnewline
\begin{minipage}[t]{0.04\columnwidth}\centering\strut
43\strut
\end{minipage} & \begin{minipage}[t]{0.21\columnwidth}\centering\strut
\QPCSymbols{\XeTeXglyph 43}\strut
\end{minipage} & \begin{minipage}[t]{0.31\columnwidth}\centering\strut
\textarabic{رضي الله عنها}\strut
\end{minipage} & \begin{minipage}[t]{0.13\columnwidth}\centering\strut
\texttt{J}\strut
\end{minipage} & \begin{minipage}[t]{0.18\columnwidth}\centering\strut
\verb$\XeTeXglyph 43$\strut
\end{minipage}\tabularnewline
\begin{minipage}[t]{0.04\columnwidth}\centering\strut
44\strut
\end{minipage} & \begin{minipage}[t]{0.21\columnwidth}\centering\strut
\QPCSymbols{\XeTeXglyph 44}\strut
\end{minipage} & \begin{minipage}[t]{0.31\columnwidth}\centering\strut
\textarabic{رضي الله عنهن}\strut
\end{minipage} & \begin{minipage}[t]{0.13\columnwidth}\centering\strut
\texttt{K}\strut
\end{minipage} & \begin{minipage}[t]{0.18\columnwidth}\centering\strut
\verb$\XeTeXglyph 44$\strut
\end{minipage}\tabularnewline
\begin{minipage}[t]{0.04\columnwidth}\centering\strut
45\strut
\end{minipage} & \begin{minipage}[t]{0.21\columnwidth}\centering\strut
\QPCSymbols{\XeTeXglyph 45}\strut
\end{minipage} & \begin{minipage}[t]{0.31\columnwidth}\centering\strut
\textarabic{رضي الله عنهما}\strut
\end{minipage} & \begin{minipage}[t]{0.13\columnwidth}\centering\strut
\texttt{L}\strut
\end{minipage} & \begin{minipage}[t]{0.18\columnwidth}\centering\strut
\verb$\XeTeXglyph 45$\strut
\end{minipage}\tabularnewline
\begin{minipage}[t]{0.04\columnwidth}\centering\strut
46\strut
\end{minipage} & \begin{minipage}[t]{0.21\columnwidth}\centering\strut
\QPCSymbols{\XeTeXglyph 46}\strut
\end{minipage} & \begin{minipage}[t]{0.31\columnwidth}\centering\strut
\textarabic{رضي الله عنهم}\strut
\end{minipage} & \begin{minipage}[t]{0.13\columnwidth}\centering\strut
\texttt{M}\strut
\end{minipage} & \begin{minipage}[t]{0.18\columnwidth}\centering\strut
\verb$\XeTeXglyph 46$\strut
\end{minipage}\tabularnewline
\begin{minipage}[t]{0.04\columnwidth}\centering\strut
47\strut
\end{minipage} & \begin{minipage}[t]{0.21\columnwidth}\centering\strut
\QPCSymbols{\XeTeXglyph 47}\strut
\end{minipage} & \begin{minipage}[t]{0.31\columnwidth}\centering\strut
\textarabic{عليه وعلى آله الصلاة والسلام}\strut
\end{minipage} & \begin{minipage}[t]{0.13\columnwidth}\centering\strut
\texttt{N}\strut
\end{minipage} & \begin{minipage}[t]{0.18\columnwidth}\centering\strut
\verb$\XeTeXglyph 47$\strut
\end{minipage}\tabularnewline
\begin{minipage}[t]{0.04\columnwidth}\centering\strut
48\strut
\end{minipage} & \begin{minipage}[t]{0.21\columnwidth}\centering\strut
\QPCSymbols{\XeTeXglyph 48}\strut
\end{minipage} & \begin{minipage}[t]{0.31\columnwidth}\centering\strut
\textarabic{عليه الصلاة والسلام}\strut
\end{minipage} & \begin{minipage}[t]{0.13\columnwidth}\centering\strut
\texttt{O}\strut
\end{minipage} & \begin{minipage}[t]{0.18\columnwidth}\centering\strut
\verb$\XeTeXglyph 48$\strut
\end{minipage}\tabularnewline
\begin{minipage}[t]{0.04\columnwidth}\centering\strut
49\strut
\end{minipage} & \begin{minipage}[t]{0.21\columnwidth}\centering\strut
\QPCSymbols{\XeTeXglyph 49}\strut
\end{minipage} & \begin{minipage}[t]{0.31\columnwidth}\centering\strut
\textarabic{عليها السلام}\strut
\end{minipage} & \begin{minipage}[t]{0.13\columnwidth}\centering\strut
\texttt{P}\strut
\end{minipage} & \begin{minipage}[t]{0.18\columnwidth}\centering\strut
\verb$\XeTeXglyph 49$\strut
\end{minipage}\tabularnewline
\begin{minipage}[t]{0.04\columnwidth}\centering\strut
50\strut
\end{minipage} & \begin{minipage}[t]{0.21\columnwidth}\centering\strut
\QPCSymbols{\XeTeXglyph 50}\strut
\end{minipage} & \begin{minipage}[t]{0.31\columnwidth}\centering\strut
\textarabic{عليهم السلام}\strut
\end{minipage} & \begin{minipage}[t]{0.13\columnwidth}\centering\strut
\texttt{Q}\strut
\end{minipage} & \begin{minipage}[t]{0.18\columnwidth}\centering\strut
\verb$\XeTeXglyph 50$\strut
\end{minipage}\tabularnewline
\begin{minipage}[t]{0.04\columnwidth}\centering\strut
51\strut
\end{minipage} & \begin{minipage}[t]{0.21\columnwidth}\centering\strut
\QPCSymbols{\XeTeXglyph 51}\strut
\end{minipage} & \begin{minipage}[t]{0.31\columnwidth}\centering\strut
\textarabic{عليهما السلام}\strut
\end{minipage} & \begin{minipage}[t]{0.13\columnwidth}\centering\strut
\texttt{R}\strut
\end{minipage} & \begin{minipage}[t]{0.18\columnwidth}\centering\strut
\verb$\XeTeXglyph 51$\strut
\end{minipage}\tabularnewline
\begin{minipage}[t]{0.04\columnwidth}\centering\strut
52\strut
\end{minipage} & \begin{minipage}[t]{0.21\columnwidth}\centering\strut
\QPCSymbols{\XeTeXglyph 52}\strut
\end{minipage} & \begin{minipage}[t]{0.31\columnwidth}\centering\strut
\textarabic{عليه السلام}\strut
\end{minipage} & \begin{minipage}[t]{0.13\columnwidth}\centering\strut
\texttt{S}\strut
\end{minipage} & \begin{minipage}[t]{0.18\columnwidth}\centering\strut
\verb$\XeTeXglyph 52$\strut
\end{minipage}\tabularnewline
\begin{minipage}[t]{0.04\columnwidth}\centering\strut
53\strut
\end{minipage} & \begin{minipage}[t]{0.21\columnwidth}\centering\strut
\QPCSymbols{\XeTeXglyph 53}\strut
\end{minipage} & \begin{minipage}[t]{0.31\columnwidth}\centering\strut
\textarabic{رحمها الله}\strut
\end{minipage} & \begin{minipage}[t]{0.13\columnwidth}\centering\strut
\texttt{T}\strut
\end{minipage} & \begin{minipage}[t]{0.18\columnwidth}\centering\strut
\verb$\XeTeXglyph 53$\strut
\end{minipage}\tabularnewline
\begin{minipage}[t]{0.04\columnwidth}\centering\strut
54\strut
\end{minipage} & \begin{minipage}[t]{0.21\columnwidth}\centering\strut
\QPCSymbols{\XeTeXglyph 54}\strut
\end{minipage} & \begin{minipage}[t]{0.31\columnwidth}\centering\strut
\textarabic{رحمهن الله}\strut
\end{minipage} & \begin{minipage}[t]{0.13\columnwidth}\centering\strut
\texttt{U}\strut
\end{minipage} & \begin{minipage}[t]{0.18\columnwidth}\centering\strut
\verb$\XeTeXglyph 54$\strut
\end{minipage}\tabularnewline
\begin{minipage}[t]{0.04\columnwidth}\centering\strut
55\strut
\end{minipage} & \begin{minipage}[t]{0.21\columnwidth}\centering\strut
\QPCSymbols{\XeTeXglyph 55}\strut
\end{minipage} & \begin{minipage}[t]{0.31\columnwidth}\centering\strut
\textarabic{رحمه الله}\strut
\end{minipage} & \begin{minipage}[t]{0.13\columnwidth}\centering\strut
\texttt{V}\strut
\end{minipage} & \begin{minipage}[t]{0.18\columnwidth}\centering\strut
\verb$\XeTeXglyph 55$\strut
\end{minipage}\tabularnewline
\begin{minipage}[t]{0.04\columnwidth}\centering\strut
56\strut
\end{minipage} & \begin{minipage}[t]{0.21\columnwidth}\centering\strut
\QPCSymbols{\XeTeXglyph 56}\strut
\end{minipage} & \begin{minipage}[t]{0.31\columnwidth}\centering\strut
\textarabic{رحمهما الله}\strut
\end{minipage} & \begin{minipage}[t]{0.13\columnwidth}\centering\strut
\texttt{W}\strut
\end{minipage} & \begin{minipage}[t]{0.18\columnwidth}\centering\strut
\verb$\XeTeXglyph 56$\strut
\end{minipage}\tabularnewline
\begin{minipage}[t]{0.04\columnwidth}\centering\strut
57\strut
\end{minipage} & \begin{minipage}[t]{0.21\columnwidth}\centering\strut
\QPCSymbols{\XeTeXglyph 57}\strut
\end{minipage} & \begin{minipage}[t]{0.31\columnwidth}\centering\strut
\textarabic{رحمهم الله}\strut
\end{minipage} & \begin{minipage}[t]{0.13\columnwidth}\centering\strut
\texttt{X}\strut
\end{minipage} & \begin{minipage}[t]{0.18\columnwidth}\centering\strut
\verb$\XeTeXglyph 57$\strut
\end{minipage}\tabularnewline
\begin{minipage}[t]{0.04\columnwidth}\centering\strut
58\strut
\end{minipage} & \begin{minipage}[t]{0.21\columnwidth}\centering\strut
\QPCSymbols{\XeTeXglyph 58}\strut
\end{minipage} & \begin{minipage}[t]{0.31\columnwidth}\centering\strut
\textarabic{مقدمة}\strut
\end{minipage} & \begin{minipage}[t]{0.13\columnwidth}\centering\strut
\texttt{Y}\strut
\end{minipage} & \begin{minipage}[t]{0.18\columnwidth}\centering\strut
\verb$\XeTeXglyph 58$\strut
\end{minipage}\tabularnewline
\begin{minipage}[t]{0.04\columnwidth}\centering\strut
59\strut
\end{minipage} & \begin{minipage}[t]{0.21\columnwidth}\centering\strut
\QPCSymbols{\XeTeXglyph 59}\strut
\end{minipage} & \begin{minipage}[t]{0.31\columnwidth}\centering\strut
\textarabic{فهرس}\strut
\end{minipage} & \begin{minipage}[t]{0.13\columnwidth}\centering\strut
\texttt{Z}\strut
\end{minipage} & \begin{minipage}[t]{0.18\columnwidth}\centering\strut
\verb$\XeTeXglyph 59$\strut
\end{minipage}\tabularnewline
\begin{minipage}[t]{0.04\columnwidth}\centering\strut
60\strut
\end{minipage} & \begin{minipage}[t]{0.21\columnwidth}\centering\strut
\QPCSymbols{\XeTeXglyph 60}\strut
\end{minipage} & \begin{minipage}[t]{0.31\columnwidth}\centering\strut
\textarabic{الفصل}\strut
\end{minipage} & \begin{minipage}[t]{0.13\columnwidth}\centering\strut
\texttt{[}\strut
\end{minipage} & \begin{minipage}[t]{0.18\columnwidth}\centering\strut
\verb$\XeTeXglyph 60$\strut
\end{minipage}\tabularnewline
\begin{minipage}[t]{0.04\columnwidth}\centering\strut
61\strut
\end{minipage} & \begin{minipage}[t]{0.21\columnwidth}\centering\strut
\QPCSymbols{\XeTeXglyph 61}\strut
\end{minipage} & \begin{minipage}[t]{0.31\columnwidth}\centering\strut
\textarabic{تمهيد}\strut
\end{minipage} & \begin{minipage}[t]{0.13\columnwidth}\centering\strut
\texttt{ }\strut
\end{minipage} & \begin{minipage}[t]{0.18\columnwidth}\centering\strut
\verb$\XeTeXglyph 61$\strut
\end{minipage}\tabularnewline
\begin{minipage}[t]{0.04\columnwidth}\centering\strut
62\strut
\end{minipage} & \begin{minipage}[t]{0.21\columnwidth}\centering\strut
\QPCSymbols{\XeTeXglyph 62}\strut
\end{minipage} & \begin{minipage}[t]{0.31\columnwidth}\centering\strut
\textarabic{تمت}\strut
\end{minipage} & \begin{minipage}[t]{0.13\columnwidth}\centering\strut
\texttt{]}\strut
\end{minipage} & \begin{minipage}[t]{0.18\columnwidth}\centering\strut
\verb$\XeTeXglyph 62$\strut
\end{minipage}\tabularnewline
\begin{minipage}[t]{0.04\columnwidth}\centering\strut
63\strut
\end{minipage} & \begin{minipage}[t]{0.21\columnwidth}\centering\strut
\QPCSymbols{\XeTeXglyph 63}\strut
\end{minipage} & \begin{minipage}[t]{0.31\columnwidth}\centering\strut
\textarabic{الباب}\strut
\end{minipage} & \begin{minipage}[t]{0.13\columnwidth}\centering\strut
\texttt{\^}\strut
\end{minipage} & \begin{minipage}[t]{0.18\columnwidth}\centering\strut
\verb$\XeTeXglyph 63$\strut
\end{minipage}\tabularnewline
\begin{minipage}[t]{0.04\columnwidth}\centering\strut
64\strut
\end{minipage} & \begin{minipage}[t]{0.21\columnwidth}\centering\strut
\QPCSymbols{\XeTeXglyph 64}\strut
\end{minipage} & \begin{minipage}[t]{0.31\columnwidth}\centering\strut
\textarabic{الجزء}\strut
\end{minipage} & \begin{minipage}[t]{0.13\columnwidth}\centering\strut
\texttt{\_}\strut
\end{minipage} & \begin{minipage}[t]{0.18\columnwidth}\centering\strut
\verb$\XeTeXglyph 64$\strut
\end{minipage}\tabularnewline
\begin{minipage}[t]{0.04\columnwidth}\centering\strut
65\strut
\end{minipage} & \begin{minipage}[t]{0.21\columnwidth}\centering\strut
\QPCSymbols{\XeTeXglyph 65}\strut
\end{minipage} & \begin{minipage}[t]{0.31\columnwidth}\centering\strut
\textarabic{تهانينا}\strut
\end{minipage} & \begin{minipage}[t]{0.13\columnwidth}\centering\strut
\texttt{`}\strut
\end{minipage} & \begin{minipage}[t]{0.18\columnwidth}\centering\strut
\verb$\XeTeXglyph 65$\strut
\end{minipage}\tabularnewline
\begin{minipage}[t]{0.04\columnwidth}\centering\strut
66\strut
\end{minipage} & \begin{minipage}[t]{0.21\columnwidth}\centering\strut
\QPCSymbols{\XeTeXglyph 66}\strut
\end{minipage} & \begin{minipage}[t]{0.31\columnwidth}\centering\strut
\textarabic{الله أكبر}\strut
\end{minipage} & \begin{minipage}[t]{0.13\columnwidth}\centering\strut
\texttt{a}\strut
\end{minipage} & \begin{minipage}[t]{0.18\columnwidth}\centering\strut
\verb$\XeTeXglyph 66$\strut
\end{minipage}\tabularnewline
\begin{minipage}[t]{0.04\columnwidth}\centering\strut
67\strut
\end{minipage} & \begin{minipage}[t]{0.21\columnwidth}\centering\strut
\QPCSymbols{\XeTeXglyph 67}\strut
\end{minipage} & \begin{minipage}[t]{0.31\columnwidth}\centering\strut
\textarabic{عز وجل}\strut
\end{minipage} & \begin{minipage}[t]{0.13\columnwidth}\centering\strut
\texttt{b}\strut
\end{minipage} & \begin{minipage}[t]{0.18\columnwidth}\centering\strut
\verb$\XeTeXglyph 67$\strut
\end{minipage}\tabularnewline
\begin{minipage}[t]{0.04\columnwidth}\centering\strut
68\strut
\end{minipage} & \begin{minipage}[t]{0.21\columnwidth}\centering\strut
\QPCSymbols{\XeTeXglyph 68}\strut
\end{minipage} & \begin{minipage}[t]{0.31\columnwidth}\centering\strut
\textarabic{سبحانه وتعالى}\strut
\end{minipage} & \begin{minipage}[t]{0.13\columnwidth}\centering\strut
\texttt{c}\strut
\end{minipage} & \begin{minipage}[t]{0.18\columnwidth}\centering\strut
\verb$\XeTeXglyph 68$\strut
\end{minipage}\tabularnewline
\begin{minipage}[t]{0.04\columnwidth}\centering\strut
69\strut
\end{minipage} & \begin{minipage}[t]{0.21\columnwidth}\centering\strut
\QPCSymbols{\XeTeXglyph 69}\strut
\end{minipage} & \begin{minipage}[t]{0.31\columnwidth}\centering\strut
\textarabic{جل جلاله}\strut
\end{minipage} & \begin{minipage}[t]{0.13\columnwidth}\centering\strut
\texttt{d}\strut
\end{minipage} & \begin{minipage}[t]{0.18\columnwidth}\centering\strut
\verb$\XeTeXglyph 69$\strut
\end{minipage}\tabularnewline
\begin{minipage}[t]{0.04\columnwidth}\centering\strut
70\strut
\end{minipage} & \begin{minipage}[t]{0.21\columnwidth}\centering\strut
\QPCSymbols{\XeTeXglyph 70}\strut
\end{minipage} & \begin{minipage}[t]{0.31\columnwidth}\centering\strut
\textarabic{جل وعلا}\strut
\end{minipage} & \begin{minipage}[t]{0.13\columnwidth}\centering\strut
\texttt{e}\strut
\end{minipage} & \begin{minipage}[t]{0.18\columnwidth}\centering\strut
\verb$\XeTeXglyph 70$\strut
\end{minipage}\tabularnewline
\begin{minipage}[t]{0.04\columnwidth}\centering\strut
71\strut
\end{minipage} & \begin{minipage}[t]{0.21\columnwidth}\centering\strut
\QPCSymbols{\XeTeXglyph 71}\strut
\end{minipage} & \begin{minipage}[t]{0.31\columnwidth}\centering\strut
\textarabic{تبارك وتعالى}\strut
\end{minipage} & \begin{minipage}[t]{0.13\columnwidth}\centering\strut
\texttt{f}\strut
\end{minipage} & \begin{minipage}[t]{0.18\columnwidth}\centering\strut
\verb$\XeTeXglyph 71$\strut
\end{minipage}\tabularnewline
\begin{minipage}[t]{0.04\columnwidth}\centering\strut
72\strut
\end{minipage} & \begin{minipage}[t]{0.21\columnwidth}\centering\strut
\QPCSymbols{\XeTeXglyph 72}\strut
\end{minipage} & \begin{minipage}[t]{0.31\columnwidth}\centering\strut
\textarabic{صلى الله عليه وسلم}\strut
\end{minipage} & \begin{minipage}[t]{0.13\columnwidth}\centering\strut
\texttt{g}\strut
\end{minipage} & \begin{minipage}[t]{0.18\columnwidth}\centering\strut
\verb$\XeTeXglyph 72$\strut
\end{minipage}\tabularnewline
\begin{minipage}[t]{0.04\columnwidth}\centering\strut
73\strut
\end{minipage} & \begin{minipage}[t]{0.21\columnwidth}\centering\strut
\QPCSymbols{\XeTeXglyph 73}\strut
\end{minipage} & \begin{minipage}[t]{0.31\columnwidth}\centering\strut
\textarabic{رضي الله عنه}\strut
\end{minipage} & \begin{minipage}[t]{0.13\columnwidth}\centering\strut
\texttt{h}\strut
\end{minipage} & \begin{minipage}[t]{0.18\columnwidth}\centering\strut
\verb$\XeTeXglyph 73$\strut
\end{minipage}\tabularnewline
\begin{minipage}[t]{0.04\columnwidth}\centering\strut
74\strut
\end{minipage} & \begin{minipage}[t]{0.21\columnwidth}\centering\strut
\QPCSymbols{\XeTeXglyph 74}\strut
\end{minipage} & \begin{minipage}[t]{0.31\columnwidth}\centering\strut
\textarabic{رضي الله عنها}\strut
\end{minipage} & \begin{minipage}[t]{0.13\columnwidth}\centering\strut
\texttt{i}\strut
\end{minipage} & \begin{minipage}[t]{0.18\columnwidth}\centering\strut
\verb$\XeTeXglyph 74$\strut
\end{minipage}\tabularnewline
\begin{minipage}[t]{0.04\columnwidth}\centering\strut
75\strut
\end{minipage} & \begin{minipage}[t]{0.21\columnwidth}\centering\strut
\QPCSymbols{\XeTeXglyph 75}\strut
\end{minipage} & \begin{minipage}[t]{0.31\columnwidth}\centering\strut
\textarabic{رضي الله عنهم}\strut
\end{minipage} & \begin{minipage}[t]{0.13\columnwidth}\centering\strut
\texttt{j}\strut
\end{minipage} & \begin{minipage}[t]{0.18\columnwidth}\centering\strut
\verb$\XeTeXglyph 75$\strut
\end{minipage}\tabularnewline
\begin{minipage}[t]{0.04\columnwidth}\centering\strut
76\strut
\end{minipage} & \begin{minipage}[t]{0.21\columnwidth}\centering\strut
\QPCSymbols{\XeTeXglyph 76}\strut
\end{minipage} & \begin{minipage}[t]{0.31\columnwidth}\centering\strut
\textarabic{رضي الله عنهما}\strut
\end{minipage} & \begin{minipage}[t]{0.13\columnwidth}\centering\strut
\texttt{k}\strut
\end{minipage} & \begin{minipage}[t]{0.18\columnwidth}\centering\strut
\verb$\XeTeXglyph 76$\strut
\end{minipage}\tabularnewline
\begin{minipage}[t]{0.04\columnwidth}\centering\strut
77\strut
\end{minipage} & \begin{minipage}[t]{0.21\columnwidth}\centering\strut
\QPCSymbols{\XeTeXglyph 77}\strut
\end{minipage} & \begin{minipage}[t]{0.31\columnwidth}\centering\strut
\textarabic{رضي الله عنهن}\strut
\end{minipage} & \begin{minipage}[t]{0.13\columnwidth}\centering\strut
\texttt{l}\strut
\end{minipage} & \begin{minipage}[t]{0.18\columnwidth}\centering\strut
\verb$\XeTeXglyph 77$\strut
\end{minipage}\tabularnewline
\begin{minipage}[t]{0.04\columnwidth}\centering\strut
78\strut
\end{minipage} & \begin{minipage}[t]{0.21\columnwidth}\centering\strut
\QPCSymbols{\XeTeXglyph 78}\strut
\end{minipage} & \begin{minipage}[t]{0.31\columnwidth}\centering\strut
\textarabic{عليه الصلاة والسلام}\strut
\end{minipage} & \begin{minipage}[t]{0.13\columnwidth}\centering\strut
\texttt{m}\strut
\end{minipage} & \begin{minipage}[t]{0.18\columnwidth}\centering\strut
\verb$\XeTeXglyph 78$\strut
\end{minipage}\tabularnewline
\begin{minipage}[t]{0.04\columnwidth}\centering\strut
79\strut
\end{minipage} & \begin{minipage}[t]{0.21\columnwidth}\centering\strut
\QPCSymbols{\XeTeXglyph 79}\strut
\end{minipage} & \begin{minipage}[t]{0.31\columnwidth}\centering\strut
\textarabic{عليه السلام}\strut
\end{minipage} & \begin{minipage}[t]{0.13\columnwidth}\centering\strut
\texttt{n}\strut
\end{minipage} & \begin{minipage}[t]{0.18\columnwidth}\centering\strut
\verb$\XeTeXglyph 79$\strut
\end{minipage}\tabularnewline
\begin{minipage}[t]{0.04\columnwidth}\centering\strut
80\strut
\end{minipage} & \begin{minipage}[t]{0.21\columnwidth}\centering\strut
\QPCSymbols{\XeTeXglyph 80}\strut
\end{minipage} & \begin{minipage}[t]{0.31\columnwidth}\centering\strut
\textarabic{عليها السلام}\strut
\end{minipage} & \begin{minipage}[t]{0.13\columnwidth}\centering\strut
\texttt{o}\strut
\end{minipage} & \begin{minipage}[t]{0.18\columnwidth}\centering\strut
\verb$\XeTeXglyph 80$\strut
\end{minipage}\tabularnewline
\begin{minipage}[t]{0.04\columnwidth}\centering\strut
81\strut
\end{minipage} & \begin{minipage}[t]{0.21\columnwidth}\centering\strut
\QPCSymbols{\XeTeXglyph 81}\strut
\end{minipage} & \begin{minipage}[t]{0.31\columnwidth}\centering\strut
\textarabic{عليهم السلام}\strut
\end{minipage} & \begin{minipage}[t]{0.13\columnwidth}\centering\strut
\texttt{p}\strut
\end{minipage} & \begin{minipage}[t]{0.18\columnwidth}\centering\strut
\verb$\XeTeXglyph 81$\strut
\end{minipage}\tabularnewline
\begin{minipage}[t]{0.04\columnwidth}\centering\strut
82\strut
\end{minipage} & \begin{minipage}[t]{0.21\columnwidth}\centering\strut
\QPCSymbols{\XeTeXglyph 82}\strut
\end{minipage} & \begin{minipage}[t]{0.31\columnwidth}\centering\strut
\textarabic{عليهما السلام}\strut
\end{minipage} & \begin{minipage}[t]{0.13\columnwidth}\centering\strut
\texttt{q}\strut
\end{minipage} & \begin{minipage}[t]{0.18\columnwidth}\centering\strut
\verb$\XeTeXglyph 82$\strut
\end{minipage}\tabularnewline
\begin{minipage}[t]{0.04\columnwidth}\centering\strut
83\strut
\end{minipage} & \begin{minipage}[t]{0.21\columnwidth}\centering\strut
\QPCSymbols{\XeTeXglyph 83}\strut
\end{minipage} & \begin{minipage}[t]{0.31\columnwidth}\centering\strut
\textarabic{رحمه الله}\strut
\end{minipage} & \begin{minipage}[t]{0.13\columnwidth}\centering\strut
\texttt{r}\strut
\end{minipage} & \begin{minipage}[t]{0.18\columnwidth}\centering\strut
\verb$\XeTeXglyph 83$\strut
\end{minipage}\tabularnewline
\begin{minipage}[t]{0.04\columnwidth}\centering\strut
84\strut
\end{minipage} & \begin{minipage}[t]{0.21\columnwidth}\centering\strut
\QPCSymbols{\XeTeXglyph 84}\strut
\end{minipage} & \begin{minipage}[t]{0.31\columnwidth}\centering\strut
\textarabic{رحمهم الله}\strut
\end{minipage} & \begin{minipage}[t]{0.13\columnwidth}\centering\strut
\texttt{s}\strut
\end{minipage} & \begin{minipage}[t]{0.18\columnwidth}\centering\strut
\verb$\XeTeXglyph 84$\strut
\end{minipage}\tabularnewline
\begin{minipage}[t]{0.04\columnwidth}\centering\strut
85\strut
\end{minipage} & \begin{minipage}[t]{0.21\columnwidth}\centering\strut
\QPCSymbols{\XeTeXglyph 85}\strut
\end{minipage} & \begin{minipage}[t]{0.31\columnwidth}\centering\strut
\textarabic{رحمهما الله}\strut
\end{minipage} & \begin{minipage}[t]{0.13\columnwidth}\centering\strut
\texttt{t}\strut
\end{minipage} & \begin{minipage}[t]{0.18\columnwidth}\centering\strut
\verb$\XeTeXglyph 85$\strut
\end{minipage}\tabularnewline
\begin{minipage}[t]{0.04\columnwidth}\centering\strut
86\strut
\end{minipage} & \begin{minipage}[t]{0.21\columnwidth}\centering\strut
\QPCSymbols{\XeTeXglyph 86}\strut
\end{minipage} & \begin{minipage}[t]{0.31\columnwidth}\centering\strut
\textarabic{رحمها الله}\strut
\end{minipage} & \begin{minipage}[t]{0.13\columnwidth}\centering\strut
\texttt{u}\strut
\end{minipage} & \begin{minipage}[t]{0.18\columnwidth}\centering\strut
\verb$\XeTeXglyph 86$\strut
\end{minipage}\tabularnewline
\begin{minipage}[t]{0.04\columnwidth}\centering\strut
87\strut
\end{minipage} & \begin{minipage}[t]{0.21\columnwidth}\centering\strut
\QPCSymbols{\XeTeXglyph 87}\strut
\end{minipage} & \begin{minipage}[t]{0.31\columnwidth}\centering\strut
\textarabic{رحمهن الله}\strut
\end{minipage} & \begin{minipage}[t]{0.13\columnwidth}\centering\strut
\texttt{v}\strut
\end{minipage} & \begin{minipage}[t]{0.18\columnwidth}\centering\strut
\verb$\XeTeXglyph 87$\strut
\end{minipage}\tabularnewline
\begin{minipage}[t]{0.04\columnwidth}\centering\strut
88\strut
\end{minipage} & \begin{minipage}[t]{0.21\columnwidth}\centering\strut
\QPCSymbols{\XeTeXglyph 88}\strut
\end{minipage} & \begin{minipage}[t]{0.31\columnwidth}\centering\strut
\textarabic{ريال}\strut
\end{minipage} & \begin{minipage}[t]{0.13\columnwidth}\centering\strut
\texttt{w}\strut
\end{minipage} & \begin{minipage}[t]{0.18\columnwidth}\centering\strut
\verb$\XeTeXglyph 88$\strut
\end{minipage}\tabularnewline
\begin{minipage}[t]{0.04\columnwidth}\centering\strut
89\strut
\end{minipage} & \begin{minipage}[t]{0.21\columnwidth}\centering\strut
\QPCSymbols{\XeTeXglyph 89}\strut
\end{minipage} & \begin{minipage}[t]{0.31\columnwidth}\centering\strut
\textarabic{أوقية}\strut
\end{minipage} & \begin{minipage}[t]{0.13\columnwidth}\centering\strut
\texttt{x}\strut
\end{minipage} & \begin{minipage}[t]{0.18\columnwidth}\centering\strut
\verb$\XeTeXglyph 89$\strut
\end{minipage}\tabularnewline
\begin{minipage}[t]{0.04\columnwidth}\centering\strut
90\strut
\end{minipage} & \begin{minipage}[t]{0.21\columnwidth}\centering\strut
\QPCSymbols{\XeTeXglyph 90}\strut
\end{minipage} & \begin{minipage}[t]{0.31\columnwidth}\centering\strut
\textarabic{شلن}\strut
\end{minipage} & \begin{minipage}[t]{0.13\columnwidth}\centering\strut
\texttt{y}\strut
\end{minipage} & \begin{minipage}[t]{0.18\columnwidth}\centering\strut
\verb$\XeTeXglyph 90$\strut
\end{minipage}\tabularnewline
\begin{minipage}[t]{0.04\columnwidth}\centering\strut
91\strut
\end{minipage} & \begin{minipage}[t]{0.21\columnwidth}\centering\strut
\QPCSymbols{\XeTeXglyph 91}\strut
\end{minipage} & \begin{minipage}[t]{0.31\columnwidth}\centering\strut
\textarabic{جنيه}\strut
\end{minipage} & \begin{minipage}[t]{0.13\columnwidth}\centering\strut
\texttt{z}\strut
\end{minipage} & \begin{minipage}[t]{0.18\columnwidth}\centering\strut
\verb$\XeTeXglyph 91$\strut
\end{minipage}\tabularnewline
\begin{minipage}[t]{0.04\columnwidth}\centering\strut
92\strut
\end{minipage} & \begin{minipage}[t]{0.21\columnwidth}\centering\strut
\QPCSymbols{\XeTeXglyph 92}\strut
\end{minipage} & \begin{minipage}[t]{0.31\columnwidth}\centering\strut
\textarabic{درهم}\strut
\end{minipage} & \begin{minipage}[t]{0.13\columnwidth}\centering\strut
\texttt{\{}\strut
\end{minipage} & \begin{minipage}[t]{0.18\columnwidth}\centering\strut
\verb$\XeTeXglyph 92$\strut
\end{minipage}\tabularnewline
\begin{minipage}[t]{0.04\columnwidth}\centering\strut
93\strut
\end{minipage} & \begin{minipage}[t]{0.21\columnwidth}\centering\strut
\QPCSymbols{\XeTeXglyph 93}\strut
\end{minipage} & \begin{minipage}[t]{0.31\columnwidth}\centering\strut
\textarabic{دينار}\strut
\end{minipage} & \begin{minipage}[t]{0.13\columnwidth}\centering\strut
\texttt{KHALID}\strut
\end{minipage} & \begin{minipage}[t]{0.18\columnwidth}\centering\strut
\verb$\XeTeXglyph 93$\strut
\end{minipage}\tabularnewline
\begin{minipage}[t]{0.04\columnwidth}\centering\strut
94\strut
\end{minipage} & \begin{minipage}[t]{0.21\columnwidth}\centering\strut
\QPCSymbols{\XeTeXglyph 94}\strut
\end{minipage} & \begin{minipage}[t]{0.31\columnwidth}\centering\strut
\textarabic{ليرة}\strut
\end{minipage} & \begin{minipage}[t]{0.13\columnwidth}\centering\strut
\texttt{\}}\strut
\end{minipage} & \begin{minipage}[t]{0.18\columnwidth}\centering\strut
\verb$\XeTeXglyph 94$\strut
\end{minipage}\tabularnewline
\begin{minipage}[t]{0.04\columnwidth}\centering\strut
95\strut
\end{minipage} & \begin{minipage}[t]{0.21\columnwidth}\centering\strut
\QPCSymbols{\XeTeXglyph 95}\strut
\end{minipage} & \begin{minipage}[t]{0.31\columnwidth}\centering\strut
\textarabic{الحزب}\strut
\end{minipage} & \begin{minipage}[t]{0.13\columnwidth}\centering\strut
\texttt{~}\strut
\end{minipage} & \begin{minipage}[t]{0.18\columnwidth}\centering\strut
\verb$\XeTeXglyph 95$\strut
\end{minipage}\tabularnewline
\begin{minipage}[t]{0.04\columnwidth}\centering\strut
96\strut
\end{minipage} & \begin{minipage}[t]{0.21\columnwidth}\centering\strut
\QPCSymbols{\XeTeXglyph 96}\strut
\end{minipage} & \begin{minipage}[t]{0.31\columnwidth}\centering\strut
\textarabic{الربع}\strut
\end{minipage} & \begin{minipage}[t]{0.13\columnwidth}\centering\strut
\texttt{ƒ}\strut
\end{minipage} & \begin{minipage}[t]{0.18\columnwidth}\centering\strut
\verb$\XeTeXglyph 96$\strut
\end{minipage}\tabularnewline
\begin{minipage}[t]{0.04\columnwidth}\centering\strut
97\strut
\end{minipage} & \begin{minipage}[t]{0.21\columnwidth}\centering\strut
\QPCSymbols{\XeTeXglyph 97}\strut
\end{minipage} & \begin{minipage}[t]{0.31\columnwidth}\centering\strut
\textarabic{الثمن}\strut
\end{minipage} & \begin{minipage}[t]{0.13\columnwidth}\centering\strut
\texttt{„}\strut
\end{minipage} & \begin{minipage}[t]{0.18\columnwidth}\centering\strut
\verb$\XeTeXglyph 97$\strut
\end{minipage}\tabularnewline
\bottomrule
\end{longtable}

\end{adjustwidth}

\end{document}